\documentclass{article}
\title{Reply to the first referee report on the draft {\it The Local Group in the Cosmic Web}}
\author{J.E. Forero-Romero \& R. Gonz\'alez}
\date{\today}
\begin{document}
\maketitle

The points raised by the referee are boldfaced. Our reply follows each point.


\section{Major Points}

\begin{itemize}

\item
{\bf One of the main aspects of the paper is the alignment between the
  three eigenvectors of the tidal tensor and the two vectors defined
  by Local Group analogues (the normal to their orbital plane and the
  line connecting the two). Figure 3 and Figure 4 show this (variants
  are shown in figure 5 \& 6 as well). What these figures lack is an
  estimation - a quantification - of the statistical significance of
  the result. This can be easily done by constructing 3xN "sister"
  samples with the same number of elements as the 2sigma, 3sigma and
  general samples, uniformly drawn in the interval [0,1] and comparing
  them. For example in the case of figure 3 \& 4 one could imagine
  plotting the 2 sigma spread in the cumulative distribution of such
  random distributions. In Fig 5 some kind of spread indicating the
  variance in the median computations could be included (along with
  the medians and spread for a uniform distribution). In Figure 6 the
  authors could also include an expectation from random
  distributions.}  

We have followed the advice from the referee. Now Figures 3, 4, 5 and
6 include a shadowed region quantifying the expectations from a random
distribution. The captions include an explanation on how the numbers
were estimated. 

Our estimation proceeds as follows. We construct a sample of 3D
vectors with the same number of elemets as the 2sigma sample. We use
the 2sigma sample to show the upper bound on the uncertainties.  

These vectors have a random distributions. This allows us to build the
integrated $\mu$ distributions. We repeat the exercise 10000 times to
compute the median together with the $5\%$ and $95\%$ percentiles for
the quantities of interest. In the case of Figures 3,4 and 6 it
corresponds to the fraction of elements with values $<\mu$; in the
case of Figure 5 it corresponds to the median value of $\mu$. 

\item
{\bf My other significant point is the smoothing scale. The study
  examines the web on a fixed scale namely ~1.4Mpc. However since the
  scale is fixed it measures different aspects of the environment for
  different LG analogues. This is explicitely shown in Fig2 left. A
  halo of mass 5e13Mdot could have a virial radius of upto 1 or
  1.3Mpc. For these haloes the smoothing is basically on the scale of
  Rvir. Whereas for haloes with masses of 5e11 Mdot the virial radius
  is probably less than 200kpc and your 1.4Mpc smoothing is equivalent
  to 7 times the virial radius.  Although I am quite sure that the
  tidal tensor eigenvectors are stable across these different scales,
  I'm not quite sure the web classification, web ellipticity or web
  prolateness are. Perhaps you should convince the reader of this
  issue.} 


We tested the robustness of our results with a smaller smoothing scale
(0.7 Mpc, from the $512^3$ grid data available on the Multidark
Database. This is mentioned in the paper. 

Concerning a larger smoothing scale we base the robustness on already
published results. In Forero-Romero et al. 2009  we presented a
through study of the environment as defined by the Tweb in terms of
the two free parameters of the model: $\lambda_{th}$ and the smoothing
scale. In that work we changed the smothing scale from 0.5 Mpc/h up to
12.5 Mpc/h and quantified the changes in the volume and mass filling
fractions (VFF and MFF) for each web environment. We found that for
the values of $\lambda_{th}\approx 2$ the VFF and MFF only change on
the order of $5\%$. In the thorough paper by Cautun et al. 2014 they
also show that the web properties are insensitive to changes of factor
of $2$ on the smoothing scale. 

We have added a short paragraph reminding the reader about these points. 
\end{itemize}

\section{Minor Issues}

\begin{itemize}
\item
{\bf The general sample spans a greater range of masses than the x-axis. In Section 3 it is given as 5e11 to 5e13, whereas the x axis in fig 2 goes from 1e12 to 1e13, please explain.}

We have extended the mass range to 13.5 in the plots.


\item
{\bf In the discussion the authors say that "the MW satellites are located at high galactic latitude, almost perpendicular to the direction of M31." I take issue wth the "almost perpendicular" comment. The plane of MW satellites is between 32 and 45 degrees from the direction to M31 depending on how it is defined. This is a small point of contention in the literature because people define it differently given the uncertainties and incompleteness limits that drive the fact that most of these galaxies are found at high galactic latitude. 
Using a fairly conservative definition (ie the 11 brightest dwarfs for which we have fairly good completeness limits) Pawlowski et al 2012 and Shaya \& Tully 2014 both find something between 32 and 45 degrees. If the authors want to stick to the "almost perpendicular" statement they will need to justify it by citation and explanation}


We have included the exact values in the Discussion section.

\item
{\bf I also want to note that the preferential accretion found by Libeskind et al 2014 shows a signal primarily along e3, but also defined as being away from e1. Indeed the accretion is somewhat planar in the e2-e3 plane. So I wouldn't say there is a contradiction.}

We have made explicit this point in the Discussion section.

\item
{\bf Lastly the discussion and interpretation of results regarding the alignment of the orbital plane with the eigenvectors could do with an engagement of the vibrant debate in the literature regarding how massive haloes have a spin flip with respect the filament or wall they are in compared to less massive haloes (e.g. Aragon-Calvo et al 2007, Codis et al 2012, Trowland et al 2013, Libeskind et al 2013, among others), likely due to major mergers.} 

We have included a paragraph on this discussion. 

\end{itemize}

\end{document}
